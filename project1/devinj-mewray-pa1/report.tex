%=====================ComS 311 LaTeX template, following the CMU 02-713 template================
%
% You don't need to use LaTeX or this template, but you must turn your homework in as
% a typeset .pdf or .doc(x)
%
% How to use:
%    1. Update your information in section "A" below
%    2. Write your answers in section "B" below. Precede answers for all 
%       parts of a question with the command "\question{n}{desc}" where n is
%       the question number and "desc" is a short, one-line description of 
%       the problem. There is no need to restate the problem. 
%       Note that "desc'' can be empty, i.e., \question{n}{} is an acceptable command.
%    3. If a question has multiple parts, precede the answer to part x with the
%       command "\part{x}".
%    4. If a problem asks you to design an algorithm, use the commands
%       \algorithm, \correctness, \runtime to precede your discussion of the 
%       description of the algorithm, its correctness, and its running time, respectively.
%    5. You can include graphics by using the command \includegraphics{FILENAME}
%
\documentclass[11pt]{article}
\usepackage{amsmath,amssymb,amsthm}
\usepackage{times,inconsolata}
\usepackage{graphicx}
\usepackage{array}
\usepackage{TiKZ}
\usepackage{framed}
\usepackage[margin=1in]{geometry}
\usepackage{fancyhdr}
\usetikzlibrary{automata,positioning}
\usetikzlibrary{arrows}
\setlength{\parindent}{0pt}
\setlength{\parskip}{5pt plus 1pt}
\setlength{\headheight}{13.6pt}
\newcommand\question[2]{\vspace{.25in}\hrule\textbf{#1. #2}\vspace{.5em}\hrule\vspace{.10in}}
\renewcommand\part[1]{\vspace{.10in}\textbf{(#1)}}
\pagestyle{fancyplain}
\lhead{\textbf{\NAME\ }}
\chead{\textbf{Report\HWNUM}}
\rhead{\textsc{Com S 311}}

\DeclareMathOperator*{\argmax}{arg\,max}
\DeclareMathOperator*{\argmin}{arg\,min}

\begin{document}\raggedright
%Section A==============Change the values below to match your information==================
\newcommand\NAME{Devin Johnson, Mason Wray}  % your name
\newcommand\HWNUM{ PA1}              % the homework number


%Section B==============Put your answers to the questions below here=======================

\question{WarWithArray}{} 

\textbf{PseudoCode:}\\
\begin{verbatim}
Input: int k, Array a[]\\
For each string e in a[]:
    For each string f in a[]:
        check if the string "e+f" is valid(e+f)
    	
    	
function valid(s):
   For each b = (k length substring in s):
       Loop through our original array a:
           check if b is in a
\end{verbatim}

\textbf{Runtime of compute2k():}\\

The algorithm essentially 4 nested loops, they run n-times, n-times, k-times, and n-times.\\
So the run-time is $O(kn^3)$


%----------------------------------
\question{WarWithBST}{}

\textbf{PseudoCode:}\\

\begin{verbatim}
Input: int k, Array a[]\\
For each string e in a[]:
    For each string f in a[]:
        check if the string "e+f" is valid(e+f)
    	
    	
function valid(s):
   For each b = (k length substring in s):
       search the BST for the substring b
\end{verbatim}


\textbf{Runtime of compute2k():}\\
The algorithm is essentially 4 nested loops, they run n-times, n-times, k-times, and log n-times.\\
So the run-time is $O(kn^2logn)$

\pagebreak
%----------------------------------
\question{WarWithHash}{}

\textbf{PseudoCode:}\\
\begin{verbatim}
Input: int k, Array a[]\\
For each string e in a[]:
    For each string f in a[]:
        check if the string "e+f" is valid(e+f)
    	
    	
function valid(s):
   For each b = (k length substring in s):
   		For each character in b
       		hash character and add to b's hash
       	look up b's hash in the HashSet
\end{verbatim}

\textbf{Runtime of compute2k():}\\
The algorithm is essentially 4 nested loops, they run n-times for the most outer loop, n-times for the second most outer loop, k-times for each substring of the 2k string, and k-times for the hash of the k-length string.\\
So the run-time is $O(k^2n^2)$


%----------------------------------
\question{WarWithRollHash}{}

\textbf{PseudoCode:}\\

\begin{verbatim}
Input: int k, Array a[]\\
    For each string e in a[]:
        For each string f in a[]:
            check if the string "e+f" is valid(e+f)
			
function valid(s):
    For each b = (k length substring in s):
        hash b and look it up in the HashSet
\end{verbatim}


\textbf{Runtime of compute2k():}\\
The algorithm is very similar to the compute2K() method in WarWithHash, except that each hash (after the first hash) is computed in O(1) time, rather than O(k) time. The rolling hash uses the last hash value to compute the new hash with one addition, one subtraction, and one multiplication, regardless of the values of n or k.\\
The algorithm is essentially 3 nested loops, they run n-times for the most outer loop, n-times for the second most outer loop, and k-times for each substring of the 2k string.\\
So the runtime is $O(kn^2).$

\end{document}
