%=====================ComS 311 LaTeX template, following the CMU 02-713 template================
%
% You don't need to use LaTeX or this template, but you must turn your homework in as
% a typeset .pdf or .doc(x)
%
% How to use:
%    1. Update your information in section "A" below
%    2. Write your answers in section "B" below. Precede answers for all 
%       parts of a question with the command "\question{n}{desc}" where n is
%       the question number and "desc" is a short, one-line description of 
%       the problem. There is no need to restate the problem. 
%       Note that "desc'' can be empty, i.e., \question{n}{} is an acceptable command.
%    3. If a question has multiple parts, precede the answer to part x with the
%       command "\part{x}".
%    4. If a problem asks you to design an algorithm, use the commands
%       \algorithm, \correctness, \runtime to precede your discussion of the 
%       description of the algorithm, its correctness, and its running time, respectively.
%    5. You can include graphics by using the command \includegraphics{FILENAME}
%
\documentclass[11pt]{article}
\usepackage{amsmath,amssymb,amsthm}
\usepackage{times,inconsolata}
\usepackage{graphicx}
\usepackage{array}
\usepackage{TiKZ}
\usepackage{framed}
\usepackage[margin=1in]{geometry}
\usepackage{fancyhdr}
\usetikzlibrary{automata,positioning}
\usetikzlibrary{arrows}
\setlength{\parindent}{0pt}
\setlength{\parskip}{5pt plus 1pt}
\setlength{\headheight}{13.6pt}
\newcommand\question[2]{\vspace{.25in}\hrule\textbf{#1. #2}\vspace{.5em}\hrule\vspace{.10in}}
\renewcommand\part[1]{\vspace{.10in}\textbf{(#1)}}
\pagestyle{fancyplain}
\lhead{\textbf{\NAME\ }}
\chead{\textbf{Report\HWNUM}}
\rhead{\textsc{Com S 311}}

\DeclareMathOperator*{\argmax}{arg\,max}
\DeclareMathOperator*{\argmin}{arg\,min}

\begin{document}\raggedright
%Section A==============Change the values below to match your information==================
\newcommand\NAME{Devin Johnson, Mason Wray}  % your name
\newcommand\HWNUM{ PA2}                      % the homework number


%Section B==============Put your answers to the questions below here=======================

\question{Data Structures used for $Q$ and $visited$}{}
We used a combination of HashMaps, Stacks, Queues, and lists.  We used a Queue first to hold the order of links that are pulled from a wiki page.  They were then popped from the Queue, checked to see if the pages contained the key words, then pushed to a stack.  We stored whether the pages contained the key words in a HashMap<String,boolean> and we also cached the wiki page in a HashMap so we did not have to scrape the page from the web again.  Once we had emptied the first Queue and filled the stack (this reversed the order so we had the first link found at the top).  Wiki links were only added to the stack if they contained the key words.  We also added the edge to the textfile for each wiki link that did contain the key words.  Now that the stack was full, we could pop each one, and then find the cached web page it was associated with in the HashMap and start the process over again with the Queue and adding each link of that page.  This continued until all links were found up to the number specified.\\

\question{Number of edges and vertices}{}
There are 200 vertices and 2861 edges

\question{Vertex with largest out degree}{}
\begin{verbatim}
/wiki/Computer_Science is the vertex with highest outDegree with a value of 199
\end{verbatim}


\question{Diameter of the Graph}{}
400

\question{Vertex/page with highest centrality}{}
\begin{verbatim}
/wiki/Computer_Science is the vertex with highest centrality with a value of 9373
\end{verbatim}


\question{Run-time analysis of GraphProcessor}{}
$V$ is the number of vertices, and $E$ is the number of edges in graph $G$.\\ 
\vspace{2mm}
\textit{outDegree(String v})\\
Runs in $O(1)$ time because you just return the size of the Linkedlist of edges for the vertex v.

\vspace{2mm}
\textit{bfsPath(String u, String v)}\\
Runs in $O(V+E)$ time because you look at each vertex and each edge once.

\vspace{2mm}
\textit{diameter()}\\
Runs in $O((V^2)V+E)$ time because you do BFS for each pair of vertices.

\vspace{2mm}
\textit{centrality(String v)}\\
Runs in $O((V^2)V+E)$ time because you do BFS for each pair of vertices.

\end{document}
